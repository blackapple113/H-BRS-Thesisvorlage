\chapter{Verwendung der \LaTeX-Klasse}
Um die Klasse als Vorlage für Dokumente zu verwenden, empfiehlt es sich, den kompletten Ordner \texttt{template} zu kopieren und entsprechend dieser Dokumentation zu verwenden. In diesem Kapitel werden die einzelnen Bestandteile des Ordners kurz erklärt. Die Verwendung der einzelnen Ordner und Dateien wird im späteren Verlauf des Dokuments näher erläutert.

\section{Bestandteile des Templates}
Die Konfiguration aller Informationen auf dem Deckblatt, sowie die Widmung können \marginpar{\texttt{assets/\\utility/*}}über die Datei \texttt{assets/utility/meta.tex} angepasst werden. Innerhalb des Ordners \texttt{assets/utility} finden sich außerdem noch eine Datei für Worttrennungen, die von \LaTeX nicht korrekt erkannt werden (\texttt{hyphenation.tex}), Akronyme (\texttt{acronyms.tex}) und Glossareinträge (\texttt{glossary.tex}).

In \texttt{assets/images} werden Bilder hinterlegt. Je nach Anzahl der Bilder im Dokument \marginpar{\texttt{assets/\\images/*}}empfiehlt es sich, diese nach Kapitel zu sortieren. Im Ordner \texttt{assets/images} befindet sich bereits ein Ordner \texttt{titlepage}, welcher die Bilder für die Titelseite enthält. Diese sollten nicht gelöscht werden, da sonst die Titelseite und somit das Dokument nicht mehr gebaut werden kann.

Der Ordner \texttt{chapter} ist dafür gedacht, die Kapitel oder Abschnitte (je nach Verwendung der Dokumentklasse (siehe \autoref{chap:klassenoptionen})) \marginpar{\texttt{chapter}}aufzuteilen und die entsprechenden Dateien dort abzulegen. Diese Aufteilung bringt den Vorteil, dass Kapitel sehr schnell neu sortiert werden können. Zusätzlich enthalten die Dateien weniger Text und sind damit übersichtlicher.

Zu einem wissenschaftlichen Dokument gehört auch Literatur. Die Informationen zu dieser \marginpar{\texttt{biblio-\\graphy.bib}}Literatur werden als BibTeX oder BibLaTeX in der Datei \texttt{bibliography.bib} hinterlegt. Ich persönlich bevorzuge es die PDF-Dokumente lokal mit abzuspeichern. Mit lokalen Dateien habe ich die Möglichkeit Anmerkungen zu machen und die Dateien mit anderer Software besser zu durchsuchen (siehe \url{https://github.com/freedmand/semantra}). Die PDF-Dokumente kommen dann in den dieser \marginpar{\texttt{literature}}Ordner \texttt{literature}. Je nach Literaturverwaltungssoftware können Einstellungen getroffen werden, um diese Dokumente direkt in der Software aufzurufen.

Für den Bau des Dokuments mit \texttt{latexmk} wird die Datei \texttt{.latexmkrc} benötigt, da sie \marginpar{\texttt{.latexmkrc}}Informationen zum Bauen des Dokumentes mit Glossar und Akronymen enthält. Wird kein Glossar- und/oder Akronymverzeichnis benötigt, empfiehlt es sich, die entsprechende Option in der Klasse zu verwenden (siehe \autoref{chap:klassenoptionen}).

Beim Kompilieren des Dokumentes entsteht ein neuer Ordner \texttt{out}. \marginpar{\texttt{out}}Dieser Ordner enthält verschiedene Kompilierungsschritte, Log- und Synchronisierungsdateien von \LaTeX. Im Zweifel kann dieser Ordner gelöscht und der Buildprozess neu gestartet werden. Die darin befindliche PDF-Datei (das gebaute Dokument) sollte jedoch vorher gespeichert werden, um einen Verlust der Daten zu vermeiden. Vorschläge für eine Konfiguration von \LaTeX-Umgebungen in verschiedenen IDEs werden auch noch vervollständigt.

Zuletzt befindet sich in \texttt{hbrs-thesis.cls} sämtliche Konfiguration für die Klassendatei. \marginpar{\texttt{hbrs-\\thesis.cls}}Diese Konfiguration kann nach Belieben angepasst werden. Sollten Grundsätzliche Änderungs- oder Verbesserungsvorschläge an dieser Datei entstehen, bitte ich darum, diese auch bei GitHub (siehe \url{https://github.com/blackapple113/H-BRS-Thesisvorlage}) einzureichen.