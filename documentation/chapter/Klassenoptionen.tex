\chapter{Optionen der Klasse hbrs-thesis}
\label{chap:klassenoptionen}
Für die Modularität der Klasse wurden einige Optionen eingebaut. Eine Pflichtoption, welche für eine korrekte Worttrennung verwendet werden muss ist die Einstellung der Sprache. Diese kann aktuell nur zwischen englisch und deutsch ausgewählt werden. Weitere Anpassungsmöglichkeiten für z.\,B. Bilinguale Dokumente müssen bisher über die Datei \texttt{hbrs-thesis.cls} manuell eingestellt werden. Die Optionen 
Die Klasse besitzt ein paar Optionen, um verschiedene Vorlieben zu befriedigen. Mithilfe der Option \texttt{classicstyle} bzw. \texttt{modernstyle} kann zwischen einer serifenlosen Schriftart und einer Schriftart mit Serifen (\url{https://de.wikipedia.org/wiki/Serife}) unterschieden werden. Serifen sollen dem Lesenden helfen die Zeilen besser zu verfolgen, um während des Lesens nicht in der Zeile zu verrutschen.

\begin{cde}
    \begin{minted}{tex}
        \documentclass[classicstyle]{hbrs-thesis}
    \end{minted}
\end{cde}